% Options for packages loaded elsewhere
\PassOptionsToPackage{unicode}{hyperref}
\PassOptionsToPackage{hyphens}{url}
%
\documentclass[
]{article}
\usepackage{amsmath,amssymb}
\usepackage{lmodern}
\usepackage{iftex}
\ifPDFTeX
  \usepackage[T1]{fontenc}
  \usepackage[utf8]{inputenc}
  \usepackage{textcomp} % provide euro and other symbols
\else % if luatex or xetex
  \usepackage{unicode-math}
  \defaultfontfeatures{Scale=MatchLowercase}
  \defaultfontfeatures[\rmfamily]{Ligatures=TeX,Scale=1}
\fi
% Use upquote if available, for straight quotes in verbatim environments
\IfFileExists{upquote.sty}{\usepackage{upquote}}{}
\IfFileExists{microtype.sty}{% use microtype if available
  \usepackage[]{microtype}
  \UseMicrotypeSet[protrusion]{basicmath} % disable protrusion for tt fonts
}{}
\makeatletter
\@ifundefined{KOMAClassName}{% if non-KOMA class
  \IfFileExists{parskip.sty}{%
    \usepackage{parskip}
  }{% else
    \setlength{\parindent}{0pt}
    \setlength{\parskip}{6pt plus 2pt minus 1pt}}
}{% if KOMA class
  \KOMAoptions{parskip=half}}
\makeatother
\usepackage{xcolor}
\usepackage[margin=1in]{geometry}
\usepackage{graphicx}
\makeatletter
\def\maxwidth{\ifdim\Gin@nat@width>\linewidth\linewidth\else\Gin@nat@width\fi}
\def\maxheight{\ifdim\Gin@nat@height>\textheight\textheight\else\Gin@nat@height\fi}
\makeatother
% Scale images if necessary, so that they will not overflow the page
% margins by default, and it is still possible to overwrite the defaults
% using explicit options in \includegraphics[width, height, ...]{}
\setkeys{Gin}{width=\maxwidth,height=\maxheight,keepaspectratio}
% Set default figure placement to htbp
\makeatletter
\def\fps@figure{htbp}
\makeatother
\setlength{\emergencystretch}{3em} % prevent overfull lines
\providecommand{\tightlist}{%
  \setlength{\itemsep}{0pt}\setlength{\parskip}{0pt}}
\setcounter{secnumdepth}{-\maxdimen} % remove section numbering
\ifLuaTeX
  \usepackage{selnolig}  % disable illegal ligatures
\fi
\IfFileExists{bookmark.sty}{\usepackage{bookmark}}{\usepackage{hyperref}}
\IfFileExists{xurl.sty}{\usepackage{xurl}}{} % add URL line breaks if available
\urlstyle{same} % disable monospaced font for URLs
\hypersetup{
  pdftitle={Relatório},
  hidelinks,
  pdfcreator={LaTeX via pandoc}}

\title{Relatório}
\author{}
\date{\vspace{-2.5em}2023-03-01}

\begin{document}
\maketitle

\hypertarget{r-code-aulas-prticas-1-e-2}{%
\section{R Code Aulas Pr?ticas 1 e 2}\label{r-code-aulas-prticas-1-e-2}}

\hypertarget{exerccio-1.1}{%
\section{Exerc?cio 1.1}\label{exerccio-1.1}}

\hypertarget{a}{%
\subsection{a)}\label{a}}

\hypertarget{entrada-de-dados}{%
\section{Entrada de Dados}\label{entrada-de-dados}}

\hypertarget{opo-1-escrever-os-dados}{%
\section{Op??o 1: Escrever os dados}\label{opo-1-escrever-os-dados}}

bicicletas\textless-c(4.3,6.8,9.2,7.2,8.7,8.6,6.6,5.2,8.1,10.9,
7.4,4.5,3.8,7.6,6.8,7.8,8.4,7.5,10.5,6.0,
7.7,8.1,7.0,8.2,8.4,8.8,6.7,8.2,9.4,7.7,
6.3,7.7,9.1,7.9,7.9,9.4,8.2,6.7,8.2,6.5)

bicicletas

\hypertarget{opo-2-ler-os-dados-de-um-ficheiro-csv-a-partir-da-diretria-de-trabalho-que-contm-o-ficheiro-ex1.csv}{%
\section{Op??o 2: Ler os dados de um ficheiro csv a partir da diret?ria
de trabalho que cont?m o ficheiro
ex1.csv}\label{opo-2-ler-os-dados-de-um-ficheiro-csv-a-partir-da-diretria-de-trabalho-que-contm-o-ficheiro-ex1.csv}}

setwd(``C:/Users/UserCarol/Dropbox/My PC
(Carol)/Documents/IST\_2022-23/R\_PE\_2022-23/Dados\_PE'') \#Fixar a
diret?ria de trabalho bicicletas1 \textless- read.csv(``ex1.csv'',
header = TRUE, sep =``;'', dec = ``.'')

head(bicicletas1) \# Permite visualizar apenas 6 linhas

\hypertarget{opo-3-ler-os-dados-de-um-ficheiro-excel-diretamente-a-partir-de-um-link-ter-internet-ligada}{%
\section{Op??o 3: Ler os dados de um ficheiro excel diretamente a partir
de um link (ter internet
ligada!)}\label{opo-3-ler-os-dados-de-um-ficheiro-excel-diretamente-a-partir-de-um-link-ter-internet-ligada}}

\#install.packages(rio) \# Este comando apenas se usa uma vez

library(rio) \# Package que permite importar dados de v?rios formatos

urlmy\textless-``\url{https://web.tecnico.ulisboa.pt/~ist13493/PE_aulas2023/R_Material_exerciciosR/ex1.xlsx}''
bicicletas2\textless-import(urlmy)

head(bicicletas2)

\hypertarget{algumas-medidas-descritivas}{%
\section{Algumas Medidas
Descritivas}\label{algumas-medidas-descritivas}}

mean(bicicletas) \# M?dia median(bicicletas) \# Mediana

\hypertarget{construir-a-funo-para-calcular-a-moda}{%
\section{Construir a fun??o para calcular a
moda}\label{construir-a-funo-para-calcular-a-moda}}

getmode \textless- function(v) \{ uniqv \textless- unique(v)
uniqv{[}which.max(tabulate(match(v, uniqv))){]} \}

moda.bic\textless- getmode(bicicletas) \# Moda print(moda.bic)
table(bicicletas) \# Confirmar a moda

quantile(bicicletas,type=2) \# Quantis
quantile(bicicletas,type=2){[}4{]}-quantile(bicicletas,type=2){[}2{]}

summary(bicicletas) fivenum(bicicletas)

var(bicicletas) \# Vari?ncia sd(bicicletas) \# Desvio Padr?o
range(bicicletas) \#M?nimo e M?ximo diff(range(bicicletas)) \# Amplitude
Amostral 100 * sd(bicicletas)/mean(bicicletas) \# Coeficiente de
Varia??o

\#install.packages(``moments'') \# Este comando deve ser usado apenas
uma vez library(moments) skewness(bicicletas) \# Coeficiente de
Assimetria kurtosis(bicicletas) \# Coeficiente de Achatamento

\hypertarget{b}{%
\subsection{b)}\label{b}}

\hypertarget{quartis-e-amplitude-interquartil-ou-inter-quantil}{%
\section{Quartis e Amplitude Interquartil (ou
inter-quantil)}\label{quartis-e-amplitude-interquartil-ou-inter-quantil}}

quantile(bicicletas,c(0.25,0.5,0.75),type=2) IQR(bicicletas)

\hypertarget{c}{%
\subsection{c)}\label{c}}

\hypertarget{quantil-amostral-de-0.68}{%
\section{Quantil Amostral de 0.68}\label{quantil-amostral-de-0.68}}

quantile(bicicletas,0.68, type=2)

\hypertarget{d}{%
\subsection{d)}\label{d}}

\hypertarget{diagrama-de-caule-e-folhas}{%
\section{Diagrama de Caule-e-Folhas}\label{diagrama-de-caule-e-folhas}}

stem(bicicletas)

\hypertarget{e}{%
\subsection{e)}\label{e}}

\hypertarget{histograma-e-caixa-de-bigodes-ou-boxplot}{%
\section{Histograma e Caixa de Bigodes (ou
Boxplot)}\label{histograma-e-caixa-de-bigodes-ou-boxplot}}

par(mfrow=c(1,2)) hist(bicicletas, col=``red'', main=````)
boxplot(bicicletas, col=''red'', main=``\,``)

\hypertarget{exerccio-1.2}{%
\section{Exerc?cio 1.2}\label{exerccio-1.2}}

\hypertarget{a-1}{%
\subsection{a)}\label{a-1}}

\hypertarget{entrada-dos-dados}{%
\section{Entrada dos Dados}\label{entrada-dos-dados}}

setwd(``C:/Users/UserCarol/Dropbox/My PC
(Carol)/Documents/IST\_2022-23/R\_PE\_2022-23/Dados\_PE'')
cotinina=read.table(``ex2.dat'',header=T) cotinina

\hypertarget{tabela-de-frequncias-absolutas-e-frequncias-relativas-fumadores-e-no-fumadores}{%
\section{Tabela de Frequ?ncias Absolutas e Frequ?ncias Relativas
(Fumadores e N?o
Fumadores)}\label{tabela-de-frequncias-absolutas-e-frequncias-relativas-fumadores-e-no-fumadores}}

classes=cotinina\(classes class(classes) faFum<-cotinina\)faFumadores
faNFum\textless-cotinina\$faNFumadores frFum\textless-faFum/sum(faFum)
frNFum\textless-faNFum/sum(faNFum)
write.table(cbind(faFum,frFum,faNFum,frNFum))

\hypertarget{b-1}{%
\subsection{b)}\label{b-1}}

\hypertarget{histogramas-de-fumadores-e-no-fumadores}{%
\section{Histogramas de Fumadores e N?o
Fumadores}\label{histogramas-de-fumadores-e-no-fumadores}}

par(mfrow=c(1,2))
barplot(faFum,names.arg=classes,main=``Fumadores'',col=``blue'',xlab=``N?vel
de Cotinina'', ylab=``Freq. Absol. Cotinina'',space=0)
barplot(faNFum,names.arg=classes,main=``N?o Fumadores'',xlab=``N?vel de
Cotinina'', col=``green'', ylab=``Freq. Absol. Cotinina'',space=0)

\hypertarget{c-1}{%
\subsection{c)}\label{c-1}}

\hypertarget{analisar-as-populaes-fumadores-e-no-fumadores-relativamente-ao-nvel-de-cotinina}{%
\section{Analisar as Popula??es Fumadores e N?o Fumadores relativamente
ao N?vel de
Cotinina}\label{analisar-as-populaes-fumadores-e-no-fumadores-relativamente-ao-nvel-de-cotinina}}

\hypertarget{exerccio-1.3}{%
\section{Exerc?cio 1.3}\label{exerccio-1.3}}

\hypertarget{a-2}{%
\subsection{a)}\label{a-2}}

\hypertarget{entrada-dos-dados-1}{%
\section{Entrada dos Dados}\label{entrada-dos-dados-1}}

rm(list=ls(all=TRUE)) \#Remove todos os objetos
mortdata\textless-read.csv(``C:/Users/UserCarol/Dropbox/My PC
(Carol)/Documents/IST\_2022-23/R\_PE\_2022-23/Dados\_PE/ex3.csv'',sep=``;'',
header=TRUE) head(mortdata) colnames(mortdata) dim(mortdata)
is.data.frame(mortdata) is.matrix(mortdata) class(mortdata)

\hypertarget{algumas-medidas-descritivas-1}{%
\section{Algumas Medidas
Descritivas}\label{algumas-medidas-descritivas-1}}

summary(mortdata)

library(``psych'') describe(mortdata) head(describe(mortdata))

mortdata5=mortdata{[}c(2,5,14,26,28,34){]} head(mortdata5)
summary(mortdata5)

\hypertarget{histogramas-e-boxplots-da-ue27-e-5-pases}{%
\section{Histogramas e Boxplots da UE27 e 5
Pa?ses}\label{histogramas-e-boxplots-da-ue27-e-5-pases}}

par(mfrow=c(2,3)) boxplot(mortdata{[},2{]}, main = ``UE27'', ylab =
``Taxa de mortalidade infantil'', col=``darkmagenta'')
boxplot(mortdata{[},5{]}, main = ``Alemanha'', ylab = ``Taxa de
mortalidade infantil'', col= ``yellow'') boxplot(mortdata{[},14{]}, main
= ``Espanha'', ylab = ``Taxa de mortalidade infantil'', col= ``red'')
boxplot(mortdata{[},26{]}, main = ``Holanda'', ylab = ``Taxa de
mortalidade infantil'', col= ``orange'') boxplot(mortdata{[},28{]}, main
= ``Portugal'', ylab = ``Taxa de mortalidade infantil'', col =``green'')
boxplot(mortdata{[},34{]}, main = ``Reino Unido'', ylab = ``Taxa de
mortalidade infantil'', col=``white'')

par(mfrow=c(2,3)) hist(mortdata{[},2{]}, main = ``UE27'', xlab = ``Taxa
de mortalidade infantil'', col=``darkmagenta'') hist(mortdata{[},5{]},
main = ``Alemanha'', xlab = ``Taxa de mortalidade infantil'', col=
``yellow'') hist(mortdata{[},14{]}, main = ``Espanha'', xlab = ``Taxa de
mortalidade infantil'', col= ``red'' ) hist(mortdata{[},26{]}, main =
``Holanda'', xlab = ``Taxa de mortalidade infantil'',col= ``orange'')
hist(mortdata{[},28{]}, main = ``Portugal'', xlab = ``Taxa de
mortalidade infantil'', col =``green'') hist(mortdata{[},34{]}, main =
``Reino Unido'', xlab = ``Taxa de mortalidade infantil'', col=``white'')

\hypertarget{c-2}{%
\subsection{c)}\label{c-2}}

\hypertarget{grfico-para-comparar-a-ue27-e-5-pases-em-1961-e-2018}{%
\section{Gr?fico para Comparar a UE27 e 5 Pa?ses em 1961 e
2018}\label{grfico-para-comparar-a-ue27-e-5-pases-em-1961-e-2018}}

dev.new()
matrix(c(mortdata{[}2,c(2,5,14,26,28,34){]},mortdata{[}59,c(2,5,14,26,28,34){]}),6,2)
matplot(matrix(c(mortdata{[}2,c(2,5,14,26,28,34){]},mortdata{[}59,c(2,5,14,26,28,34){]}),6,2),type=``l'',
ylab=``Taxa m?dia de mortalidade infantil'',xlab=``UE27 e 5 Pa?ses'',
col=c(``blue'',``green''),axes=FALSE)
legend(``topleft'',c(``1961'',``2018''),pch = ``--'', col = c(``blue'',
``green''), bty = ``n'')
axis(1,1:6,c(``UE27'',``DE'',``ES'',``NL'',``PT'',``UK''))
axis(2,c(0,20,40,60,80,100))

\hypertarget{d-1}{%
\subsection{d)}\label{d-1}}

\hypertarget{algumas-medidas-amostrais-relativas-a-31-pases-em-2018}{%
\section{Algumas Medidas Amostrais relativas a 31 Pa?ses em
2018}\label{algumas-medidas-amostrais-relativas-a-31-pases-em-2018}}

mortdata{[}59,-c(1:4){]} \#Retira 4 colunas e ficam 31 colunas
apply(mortdata{[}59,-c(1:4){]},1,mean)
apply(mortdata{[}59,-1{]},1,median) apply(mortdata{[}59,-1{]},1,var)
apply(mortdata{[}59,-1{]},1,sd)/apply(mortdata{[}59,-1{]},1,mean)
(apply(mortdata{[}59,-1{]},1,sd)/apply(mortdata{[}59,-1{]},1,mean))*100

\hypertarget{e-1}{%
\subsection{e)}\label{e-1}}

\hypertarget{anlise-da-evoluo-temporal-da-taxa-de-mortalidade-infantil-em-portugal}{%
\section{An?lise da Evolu??o Temporal da Taxa de Mortalidade Infantil em
Portugal}\label{anlise-da-evoluo-temporal-da-taxa-de-mortalidade-infantil-em-portugal}}

plot(mortdata{[},1{]},mortdata\(PT, main="Evolu??o da Mortalidade Infantil em Portugal",  ylab="Taxa M?dia", xlab="Anos",col="red", pch=16)# type="l") points(mortdata[,1],mortdata\)UE27,
col=``blue'', pch=16) leg\_cols \textless- c(``red'', ``blue'') leg\_sym
\textless- c(16, 16) leg\_lab \textless- c(``Portugal'', ``UE27'')
legend(x = ``topright'', col = leg\_cols, pch = leg\_sym, legend =
leg\_lab, bty = ``n'')\#bty=``n'' retira a ``moldura'' da legenda

\hypertarget{f}{%
\subsection{f)}\label{f}}

\hypertarget{entrada-dos-dados-taxa-de-mortalidade-infantil-e-o-pib}{%
\section{Entrada dos Dados (Taxa de Mortalidade Infantil e o
PIB)}\label{entrada-dos-dados-taxa-de-mortalidade-infantil-e-o-pib}}

pib2018\textless-read.table(``C:/Users/UserCarol/Dropbox/My PC
(Carol)/Documents/IST\_2022-23/R\_PE\_2022-23/Dados\_PE/ex3f.dat'',
header=TRUE)\$TaxaPIB

\#Estudo da Associa??o Linear entre as Duas Vari?veis

mortdata\(PT mortdata\)PT{[}-c(1,60){]} \#Retira a informa??o do ano de
2019 para Portugal
plot(pib2018,mortdata\(PT[-c(1,60)],xlab="PIB: Taxa de Crescimento Real",  ylab="Taxa M?dia de Mortalidade Infantil (1961 a 2018)") cor(pib2018,mortdata\)PT{[}-c(1,60){]})\#
Coeficiente de Correla??o Linear de Pearson

\end{document}
